\documentclass{article}

\usepackage[colorlinks=true, allcolors=blue]{hyperref}
\usepackage[letterpaper,top=2cm,bottom=2cm,left=3cm,right=3cm,marginparwidth=1.75cm]{geometry}
\usepackage[T1]{fontenc}
\usepackage[polish]{babel}
\usepackage[utf8]{inputenc}
\usepackage{amsmath}
\usepackage{autobreak}
\usepackage{graphicx}

\title{Praca Licencjacka - Projekt Frostbite}
\author{Paweł Olszewski, Mateusz Maćkowiak, Kacper Kruger, Maksymilian Keller}
\date{Styczeń 2022 - Czerwiec 2023}
\begin{document}

\maketitle
\tableofcontents
\newpage
\section{Opis problemu}
\subsection{Cel projektu}
Celem projektu jest stworzenie oryginalnej gry survival z widokiem izometrycznym. Dzięki niej będzie możliwe zbadanie możliwości zaaplikowania algorytmów pathfindingu w losowowo-wygenerowanym labiryncie na kształt którego ma wpływ użytkownik programu. Sprawdzenie potencjału wykorzystania proceduralnych szumów (ang. Procedural Noise) do usprawnienia procesu tworzenia świata przedstawionego w grze. Może to doprowadzić do rozwoju dziedziny zajmującej się symulacjami komputerowymi. Gracz ma za zadanie przetrwać w nieprzyjaznym świecie, zbierająć bogactwo z przedmiotów do zebrania w grze, przy losowo wygenerowanej mapie, za pomocą wcześniej przedstawionych środków. Cała rozgrywka opiera się na poruszaniu się postacią tak, aby mogła przeżyć w trudnych warunkach. Trzeba zadbać o jej jedzenie i zdrowie. Planować swoje akcje na przyszłość, np. zebrać odpowiednią ilość pożywienia na noc, odpowiednią ilość opału aby oświetlić swoje otoczenie w ciemnych lokalizacjach, a także unikać niebezpiecznych miejsc, nie mając odpowiedniego ekwipunku, żeby pokonać znajdujących się w niej przeciwników. Gra ma przyciągnąć jak największą ilość użytkowników. Dzięki temu powstanie wiele różnych rozgrywek, na podstawie których będzie można wyciągać w przyszłości więcej informacji na temat efektywności wykorzystanych algorytmów i potencjału projektu do zarobku komercyjnego.
\newpage
\subsection{Porównanie do obecnych na rynku gier typu survival z widokiem izometrycznym}
\subsubsection{Don't Starve}
Producent	Klei Entertainment\\
Kompozytor	Vincent de Vera, Jason Garner\\
Wersja	All's Well that Maxwell / 22 października 2013\\
Data wydania	Windows, Linux, macOS: 23 kwietnia 2013 PlayStation 4: 7 stycznia 2014 Xbox One: 26 sierpnia 2015\\
- \textbf{Wspólne Własności} \\
- Widok z rzutu izometrycznego\\
- Wymógi zaspokojenia podstawowych potrzeb postaci, np. Wskaźnika Głodu\\
- Grafika 2D\\
- Zbieranie przedmiotów, ekwipunek\\
- Gra typu survival\\
- Losowo Wygenerowany Świat\\
- Przeróżne Biomy\\
- Przeróżni Przeciwnicy\\
- Cykl dnia i nocy\\
- \textbf{Różne Własności} \\
- Zawartość pixel art'u w grze\\
\subsubsection{Terraria}
Producent	Re-Logic\\
Wydawca	Re-Logic 505 Games Spike Chunsoft (Japonia)\\
Projektant	Andrew Spinks\\
Wersja	1.4.4.9\\
- \textbf{Wspólne Własności} \\
- Grafika 2D\\
- Zbieranie przedmiotów, ekwipunek\\
- Gra typu survival\\
- Przeróżne Biomy\\
- Przeróżni Przeciwnicy\\
- Losowo Wygenerowany Świat\\
- Zawartość pixel art'u w grze\\
- Cykl dnia i nocy\\
- \textbf{Różne Własności} \\
- Widok z rzutu izometrycznego\\
- Wymógi zaspokojenia podstawowych potrzeb postaci, np. Wskaźnika Głodu\\
\newpage
\subsection{Potencjalne przyszłe zastosowania}
Każda osoba, którą gra wciągnie najprawdopodobniej wspomni o niej koleżance, koledze, znajomemu, a ci następnym. Dzięki przyciągnięciu jak największej ilości graczy i pozytywnym jej odebraniu, gra nabeira na popularności, którą można spieniężyć.
\subsubsection{Reklamy}
Poszukanie potencjalnych reklamodawców, którzy pragnęliby zwiększyć popularność własnych produktów. Każda impresja wywołana reklamą u gracza, posiadałaby swój koszt.
\subsubsection{Płatność za dostęp do gry}
Aby gracz mógł zagrać w pełną wersję gry, powinien zapłacić uprzednio jej cenę.
\subsubsection{Część pakietu gier dostępnego w przypadku zapłacenia subskrypcji}
Wiele sklepów oferuje miesięczną subskrypcję za dostęp do gry. Zawarcie umowy z jednym z takich klientów umożliwiłoby dołączenie gry do jednego z takich pakietów.
\subsubsection{Kod jako podstawa innych gier typu survival}
Dzięki kodowi stworzonej gry, zostanie ułatwione tworzenie podobnych gier typu survival o innej tematyce przy zachowaniu istniejących mechanik. Na przykład, symulatora astronauty próbójącego przeżyć w kosmosie. Pomysłów może być wiele.
\newpage
\subsection{Źródła}
\href{http://physbam.stanford.edu/cs448x/old/Procedural_Noise_FAQ.html}{Proceduralne Szumy (ang. Procedural Noise) - Strona Internetowa Uczelni Stanford}
\begin{verbatim}
    http://physbam.stanford.edu/cs448x/old/Procedural_Noise_FAQ.html
\end{verbatim}\\
\href{https://pl.wikipedia.org/wiki/Don%E2%80%99t_Starve}{Gra "Don't Starve Together" - Wikipedia}\\
\begin{verbatim}
    https://pl.wikipedia.org/wiki/Don%E2%80%99t_Starve
\end{verbatim}\\
\href{https://store.steampowered.com/app/322330/Dont_Starve_Together/}{Gra "Don't Starve Together" - Strona Sklepu "Steam"}\\
\begin{verbatim}
    https://store.steampowered.com/app/322330/Dont_Starve_Together/
\end{verbatim}\\
\href{https://pl.wikipedia.org/wiki/Terraria}{Gra "Terraria" - Wikipedia}\\
\begin{verbatim}
    https://pl.wikipedia.org/wiki/Terraria
\end{verbatim}\\
\href{https://store.steampowered.com/app/105600/Terraria/}{Gra "Terraria" - Strona Sklepu "Steam"}\\
\begin{verbatim}
    https://store.steampowered.com/app/105600/Terraria/
\end{verbatim}\\
\newpage
\section{Projekt i analiza}
%Póki co gra nie jest jeszcze dokończona, dlatego w tej sekcji nie ma diagramów. Ale jest jakiś zarys. Ten zapis zostanie usunięty w przyszłych wersjach dokumentu.
\subsection{Aktorzy i Przypadki użycia}
%Gracz
\subsection{Wymagania funkcjonalne i niefunkcjonalne}
\subsection{Diagram klas}
\subsection{Diagram modelu danych}
\subsection{Projekt interfejsu użytkownika}
%Tu będą obrazki gotowego interfejsu. I opis obrazków.
\includegraphics[scale=0.5]{UIMainMenu.png}
\section{Implementacja}
\subsection{Architektura rozwiązania}
\subsection{Użyte wzorce projektowe}
%Singleton
\subsection{Diagramy uwzględniające architekturę całości}
\subsection{Użyte technologie}
%package'e, biblioteki itp.
\section{Testy}
\subsection{Testy Manualne}
\subsection{Testy Jednostkowe}
\subsection{Testy Integracyjne}
\subsection{Testy Wydajnościowe}
\subsection{Raport}
\section{Podział prac w projekcie}
\subsection{Paweł Olszewski}
\subsection{Mateusz Maćkowiak}
\subsection{Kacper Kruger}
\subsection{Maksymilian Keller}
\section{Bibliografia}
\end{document}